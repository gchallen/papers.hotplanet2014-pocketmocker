\section{Feasibility}
\label{sec-feasibility}

A natural question to ask about data mocking is if a system was developed
that provided this feature, could it be deployed to a large number of users?
We believe that the answer is yes, at least on Android smartphones. Below we
briefly discuss two ways to mock apps.

\subsection{Platform Support}

One way to implement mocking is to add support to smartphone platforms. For
example, when an app requested the device's current location from a platform
service, the service could either return the user's real location or a mocked
location. Given the locked-down nature of major smartphone platforms today,
this means that mocking would require the cooperation of the companies such
as Google, Apple, and Microsoft that maintain the dominant three smartphone
platforms available today. Unfortunately, we expect that all smartphone
platform providers have an interest in perpetuating the exposure of personal
information to apps that data mocking is intended to frustrate, making them
unwilling to implement this feature.

\sloppypar{Another option is to use an open-source platform, which would
limit mocking to the 80\% of smartphones worldwide running
Android~\cite{android-marketshare}. Several previous projects with similar
goals including AppFence~\cite{droids-ccs11},
MockDroid~\cite{mockdroid-hotmobile11}, and Android record and
replay~\cite{recordreplay-hotmobile11} have shown the feasibility of this
approach through platform modifications, although neither implemented mocking
as we have described it. However, while utilizing Android would allow
platform changes required to implement mocking to be implemented, deploying
them to users would still be challenging. We expect that even if they are
interesting in mocking, few users are willing or able take the steps required
to replace the built-in platform software---commonly referred to as
``rooting'' or ``jailbreaking''.}

There are two potential ways around this roadblock. First, mocking could be
integrated into popular alternate Android platform distributions such as
CyanogenMod. While this community is small, they may be disproportionately
interested in mocking given their willingness to void their warranties and
the intentions of smartphone device manufactures. Even a small amount of
mocking could lead the smartphone privacy conversation in the right
direction. Second, at some point smartphones may be required to provide more
configurability at the platform level to support apps, similar to the way
that desktop operating systems allow apps to install device drivers, thus
potentially opening the door for widespread distribution of data mocking.

\subsection{App Rewriting}

A more promising alternative that avoids the need to modify the underlying
smartphone platform is to utilize the ability to recover the source of
Android APKs through decompilation. Access to the resulting source files
would allow rewriting API calls to return mocked data, producing a mockable
version of the original app. While decompilation has limitations, we are
excited by this technique and actively exploring this approach.
