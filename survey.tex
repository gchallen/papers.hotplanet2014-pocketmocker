
\begin{table}[t]
  
  
\begin{tabularx}{\columnwidth}{Xcccccc}
& \multicolumn{6}{c}{\normalsize{\textbf{Attribute Count}}} \\
& {\normalsize{\textbf{0}}} 
& {\normalsize{\textbf{1}}} 
& {\normalsize{\textbf{2}}} 
& {\normalsize{\textbf{3}}} 
& {\normalsize{\textbf{4}}} 
& {\normalsize{\textbf{5}}} \\ \midrule
Unreasonable Fear & 48&33&16&2&0&0 \\
Uncomfortable & 24&19&18&16&12&11 \\
Interested in changing & 18&22&19&21&11&10 \\

\end{tabularx}

  
  \caption{\textbf{Summary of survey results.} Aggregates are shown for the
    three specific questions addressed in Section~\ref{subsec-surveyresults}.
    All values are percentages.}
  
  \label{table-surveysummary}

\end{table}

\section{Survey}
\label{sec-survey}

To gauge interest in mocking we distributed an IRB-approved survey to
students, faculty and staff the University at Buffalo. No incentives were
provided for completing the survey, and all respondents were required to
indicate consent before proceeding to the questions. Over four days, we
recorded 91 responses.

\subsection{Questions}

Table~\ref{table-surveyquestions} summarizes the survey we distributed. It
had three parts, each consisting of questions concerning five personal
attributes: home location, weight, activity level, income level, and
sociability. The goal of the first part was to assess how aware respondents
were of the information smartphones apps could collect about them.
Respondents were instructed to assume that the hypothetical app had been
installed and granted the permissions it requested. The goal of the second
part of the survey was to assess how comfortable respondents were with the
data smartphone apps could collect about them. Finally, the third part
assessed interest in mocking by determining how interested respondents were
in misleading apps about the five attributes.

We intentionally chose a range of attributes. We considered home address and
social network as straightforward to determine for an app with the right
permissions, in this case the ability to track user's location (home address)
or observe who they communicate with (social network). At the time we
considered activity level to be more difficult to determine, although now
that activity recognition has been integrated into widely-available libraries
such as Google Play Services this may be much simpler to measure using the
accelerometer. Finally, we chose two attributes that we were not sure could
actually be determined by smartphones: income and weight.

Based on the capabilities of current smartphones we classified answers to the
accuracy questions as either reasonable or unreasonable. As an example, we
considered it unreasonable that an app could know a user's income to within
\$1 / year or their weight to within 1 lb. However, it is possible that by
using the accelerometer and studying a user's gait their weight could be
estimated, and the widespread adoption of smartphone-based payment systems
such as Google Wallet along with socioeconomic map-matching may make income
levels estimable soon. So even the answers we marked as unreasonable may not
remain so for long.

\subsection{Results}
\label{subsec-surveyresults}

Table~\ref{table-surveyresults} shows detailed results of our mocking survey.
When analyzing the results, we were interested in three questions matching
the three sections of our survey. First, how reasonable were respondents
fears about information that apps might be able to determine? Second, how
comfortable were they sharing data with apps? And finally, were respondents
interested in modifying the data-driven impressions app might form of them?
Table~\ref{table-surveysummary} reports aggregate results relevant to these
three questions.

First, we found that respondents to be reasonably suspicious of what apps
might know about them, with 52\% indicating that an app might know at least
one personal attribute to a level that we marked as unreasonable today but
only 18\% indicating that apps might know two attributes of unreasonable
levels. The two unreasonable attributes most-frequently reported as knowable
by respondents were their income to \$100 / year (24\%), and the quantity and
type of the exercise they engaged in (20\%). Overall, users' intuition about
what attributes were easy to determine and what were hard matched ours,
reflected by the accuracy percentages in the table. No users thought an
app would not be able to determine anything about their home address,
whereas 39\% didn't think an app could determine anything about their weight.

Second, our survey showed that many respondents were uncomfortable with
smartphones knowing these aspects of their personal lives. Only 24\% were
comfortable, defined as a score of 2 or above on the 1--5 scale, with all
five attributes, and a majority (57\%) were uncomfortable with two or more.
Our results match the privacy concerns reported by smartphone users to other
surveys~\cite{truste-privacy}. Reported comfort levels on individual
attributes were also interesting, with users seeming the least comfortable
with smartphones knowing their home address---which is possible---and their
income---which, at least today, may not be. Comfort levels regarding
knowledge of a users social network were evenly distributed.

Finally, when asked about mocking, of the 91 users that completed the survey,
82\% wanted to mock at least one attribute and 60\% wanted to mock two, with
mocking users requesting an average of 2.6 mocking attributes each. Interest
in mocking different attributes was well distributed, with the percentage of
mocking responses per attribute varying from a low of 26\% for activity level
to a high of 66\% for income.

Unsurprisingly, users were most interested in mocking attributes that they
were uncomfortable with their smartphone knowing, such as home address and
income level. Surprisingly, most users seemed to want to appear to make
\textit{less} money than they actually do, which is not what we expected. We
speculate that this may be because users believe that they will see fewer ads
if advertisers believe that they are poor. In any case, it shows that it may
be difficult to determine what changes users see as desirable.

Overall, however, the results lead us to the conclusion that smartphone
users:
\begin{itemize}

\item have reasonable expectations about what smartphone apps might be able to
learn about them,
\item are uncomfortable with apps knowing these things,
\item and are interested in misleading apps.

\end{itemize}

\begin{table}[t]

{\small
\begin{framed}
  \textbf{What can data collected by your smartphone reveal about you?}

  {\small The questions below assess how much you think your phone is able to
    determine about you without asking. All the questions assume that you
    have installed the application and granted it the permissions it
  requested.}

  \begin{itemize}[leftmargin=15pt,noitemsep]
    \item \uline{Without asking, what could an application determine about} your
      yearly income?
      \begin{itemize}[leftmargin=15pt,noitemsep]
        \item An application could predict my income exactly, to within \$1 /
          year.
        \item An application could predict my income to within \$100 / year.
        \item An application could predict my income to within \$10,000 / year.
        \item An application could not determine anything about my income
          level.
      \end{itemize}
    \item $\ldots$ your weight?
    \item $\ldots$ your social network?
    \item $\ldots$ your activity level?
    \item $\ldots$ where you live?
  \end{itemize}


\end{framed}

\begin{framed}
  \textbf{What are you comfortable with applications knowing about you?}

  {\small The questions below assess how comfortable you are with smartphone
    application being able to determine the same things about you we asked
    about in the first section. \textbf{Please indicate your comfort level
  between not comfortable at all (1) and completely comfortable (5).}}

  \begin{itemize}[leftmargin=15pt,noitemsep]
    \item \uline{How comfortable are you with smartphone applications
      knowing} your yearly income?
    \item $\ldots$ your weight?
    \item $\ldots$ about your social network?
    \item $\ldots$ your activity level?
    \item $\ldots$ where you live?
  \end{itemize}

\end{framed}

\begin{framed}
  \textbf{Would you like to alter what your smartphone knows about you?}

  {\small These questions assess your interest in altering what your
    smartphone knows about you. Assume that a system exists that would allow
  you to change the qualities as indicated by the questions.}

  \begin{itemize}[leftmargin=15pt,noitemsep]
    \item \uline{If a smartphone application could accurately determine} my income
      level 
      \begin{itemize}[leftmargin=15pt,noitemsep]
        \item I would like to appear to have a lower yearly income than I
          actually do.
        \item I would like to appear to have a higher yearly income than I
          actually do.
        \item I am comfortable revealing my true income level to the
          application.
      \end{itemize}
    \item $\ldots$ my weight
    \item $\ldots$ my social network
    \item $\ldots$ my activity level
    \item $\ldots$ where I live
  \end{itemize}

\end{framed}
}


\caption{\textbf{Mocking survey questions.} Respondents were asked three
groups of questions about five aspects of their personal lives their
smartphone could observe. For each group one sample question and answers is
shown.}

  \label{table-surveyquestions}
\end{table}
