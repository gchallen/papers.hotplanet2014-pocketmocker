\begin{abstract}

Smartphones represent the most serious threat to user privacy of any
widely-deployed computing technology. Unfortunately, existing permission
models provide smartphone users with limited protection, in part due to the
difficulty users have distinguishing between legitimate and illegitimate use
of their data. A mapping app may upload the same location information it uses
to download maps (legitimate) to a marketing agency interested in delivering
location-based ads (illegitimate). However, armed with the right technology
users can turn apps' interest in personal data against them by intentionally
manipulating the data that they expose. We refer to the intentional
substitution of real data with artificial data intended to alter an apps
perception of a user as \textit{mocking} to differentiate this approach from
other privacy-motivated techniques that focus on concealing data. In this
paper, we explore the desirability and implications of this approach,
present results from a survey suggesting that many users are interested in
mocking apps, and discuss ethical and practical issues related to widespread
app mocking.

\end{abstract}
