\section{Discussion}
\label{sec-discussion}

Even if feasible and desirable mocking raises a set of unique ethical
questions related to the relationship between smartphone users and apps,
discussed below.

\subsection{Is Mocking Cheating?}

A common reaction by many to mocking is to conflate it with cheating, given
that it involves misleading apps about a users true nature. This response,
however, begs the question: what are the rules of the game? Cheating must be
defined with respect to an agreement, and we are not convinced that users
have actually agreed to provide an unlimited amount of information about
their personal lives to smartphone apps. While installing an app does involve
allowing it to access certain information, we believe that it is reasonable
for users to expect that apps request information required by the features
they provide and use it only to provide those features. Unfortunately,
particularly once the information leaves the device, users quickly lose
control over their data.

Others point out that mocking likely violates the terms of service (TOS) that
app users are frequently required to agree to during the installation
process. We have not yet performed the detailed examination of common
smartphone app TOS required to determine whether this is true, but would be
surprised if TOS agreements could be written in ways preventing users from
misleading apps. The reason is that there is considerable overlap between
mocking behaviors and things that users might legitimately do in order to
alter how smartphone apps perceive them. For example, can an app TOS prevent
a user from leaving their smartphone home during a night out? Require that
users keep their smartphones on their person all the time? Prevent a user
from loaning their device to a friend for a period of time? Or require that a
user perform all of their smartphone interaction with the same device? The
natural answer to these questions seems to be no, but this begs the question
of whether there is a meaningful difference between actually leaving the
device at home or bringing it but mocking apps into thinking it is at home.

Another common objection is that the current smartphone app ecosystem depends
on collecting user information in order to subsidize app development, most
commonly by using personal information to embed targeted advertisements in
apps, and that users benefit from lower app prices as a result. There are two
problems with this argument. First, as mentioned earlier smartphone users
have indicated on surveys that they are not comfortable with this model of
subsidizing apps through personal data collection. Second, believing that
this model is really a good deal for smartphone users requires them to trust
the same companies that are actively trying to monetize this information. A
sense that they are not receiving adequate compensation for their information
may drive user discomfort with this business model. In any case, because
smartphone users have different privacy concerns and expectations, not every
user should be required to trade data for service.

\subsection{Is Mocking Safe?}

A more serious concern with mocking concerns the effect it might have on
apps, particularly ones that are health-related. If mocking confuses an app
designed to remind a user to take a pill, it could have serious health
consequences. Here it is important to distinguish between the possible
side-effects of mocking on legitimate apps and the intentional effect of
mocking on apps that the user is intending to mislead. For example, if a
doctor asks a patient who wants to get fit to install a pedometer to help
increase their activity level and the patient chooses to mislead this app
with mocked activities, then the main problem is not really the mocking
feature. If the user wants the app to help them become healthier, they will
cooperate; if not, they can always refuse to be monitored altogether.

We believe that remaining safety concerns can be addressed through careful
system design. Mocking systems should allow users to configure which apps to
mock to avoid mocking safety-critical apps. It may also be helpful to allow
apps to issue explicit requests to not be mocked which users could approve or
ignore as a reminder to adjust mocking settings on a per-app basis.

\subsection{What Effect Would Mocking Have?}

Finally, we consider the effect that data mocking would have on smartphone
data collection and privacy if deployed on a significant number of devices.
First, we would expect to see an interest among app developers in deploying
countermeasures to detect or eliminate mocked data. While single-app attacks
may be defeated by carefully engineering the mocking system to provide
consistent false data, a more difficult or impossible set of attacks are
launched by colluding with other devices or with surrounding infrastructure.
Interdevice collusion is more feasible, since it could be launched by
cooperating instances of the same app. Foiling these attacks might require
mocking cross-device interaction to fool the local app, or simply disabling
interfaces such as Bluetooth allowing device-to-device communication.

Collusion with the infrastructure would represent a more serious challenge to
data mocking. As an example, if mobile data networks began reporting
smartphone user's location directly to app providers apps could use this
information to pierce the mocking context by comparing the location being
reported to them by the smartphone to infrastructure-reported location. While
this type of collusion is the most effective way to shut down our mocking
approach, it would also represent an unprecedented level of cooperation
between network providers and the companies selling apps and services. We
anticipate that these types of agreements would be highly-unattractive to
smartphone users already concerned about their privacy.

\newpage

Our ultimate hope in discussing smartphone mocking is to
initiate a conversation about what our personal data is worth. Today, because
smartphone users lack effective tools to control the data they provide to
apps, they are effectively surrendering their personal information without
receiving anything in return. So while this data is clearly worth something,
as evidenced by advertisers scrambling to develop novel location-based
analytics, as long as we give it away for free we will never know how much.
If mocking causes apps to begin to be suspicious of the personal data they
can collect, this may help make legitimate information about users even more 
valuable.
